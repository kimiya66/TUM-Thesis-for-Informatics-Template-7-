\chapter{\abstractname}
\acrlong{ap} is an interesting field of research in autonomous driving because parking, and especially parallel parking, has always been a challenge for drivers, even experts. Although study on autonomous parking has been ongoing for more than a decade, it is still open for new ideas and researchers because most of the proposed methods are not completely practical or suitable to apply on real vehicles on public streets. When we speak about autonomous parking, we always confront two research aspects. First is parking detection and research on algorithms to separate vacant and occupied parking spots. The second is path planning and motion generation algorithms to perform parking maneuvers. This thesis presents an implementation of a parallel parking algorithm. For this, a complete parking scenario which includes all parallel parking steps from finding a parking place to controlling the vehicle to move it into the detected parking place. For the parking vacancy detection step, the ACFObjectDetector from the Matlab Computer Vision Toolbox was, used and for the motion generation step a parallel parking algorithm for a non-holonomic vehicle has been implemented. The motion planning algorithm is based on distance measurements with respect to the vehicle's environment and surrounding objects. The approach has been implemented on Carla simulator by using python as \acrshort{api}.


