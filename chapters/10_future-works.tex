% !TeX root = ../main.tex
\chapter{Future Work}\label{chapter: future_works}
This chapter presents all of the possible improvements of this simulation in the future. The goal is to use this method as a practical and reliable way of parking on a real steer without help of driver.
\section{Design a new sensor to improve range measurements}
Sensors are the main problem as it has been mention in the last chapter \ref{chapter:limitations} which are not giving perfect measurement result. In order to make this parking maneuver algorithm practical before the deadline of master thesis, best solution was to choose between current sensors offered by Carla simulator. This was working fine but for making better result and more accurate range measurement, a better sensor should be used. One idea would be to design a new sensor which can also be interface between different sensors in Carla for instance a sensor could be defined here which would be just attached to the ego vehicle and find a detected vehicle and from the id of detected vehicle find its location and the coordinate of each edge of the vehicle to get a precise distance.
we could also use Lidar sensor here and fix the problem of it to show 3D visual results. 
Other solution is to get the sensors from other applications and find a way to connect it to Carla or to the Lidar channels that are currently presented by Carla. So a new sensor should be designed in future development.
\section{Improve maneuver algorithm}
As already pointed out in the previous chapters, the idea of this maneuver algorithm is to implement iteratively to get the better results. It can be also seen in real parallel parking in street(not garage parking) that when driver wants to park in a small parking place, there should be some iterative forward/backwards movement of the car to get the perfect position in the parking place. By the first maneuver(as we implemented here), vehicle enters to the parking place but its position in the parking is not always perfect. That may be parked crooked or may not be positioned as the same line as other adjacent vehicles. Hence, to make a better position, in the next maneuver it should move forward and get the opposite direction for movement(velocity signs changed from -v to +v for moving forward) and also opposite direction of steering angle(from $-\phi$ to $+\phi$ and vise versa). This movements will be repeated till the vehicle gets a perfect place in the parking-bay as a driver always does in parallel-parking. So another ongoing improvement is to make the maneuver iterative to contain forward movements besides backward movement. To this end, a perfect final constraint for the end of this iteration should be defined. This constraint would define the end position of the vehicle in the parking place. Sensors are also required here to measure precise distances to rear and front cars to decide for final location of vehicle in parking-bay. Number of iteration depends on the ego vehicle's size as well as parking's area. 
\section{test in other simulator}
Another idea is to test this method in other simulators to see how algorithm would be worked in different scenarios and use various tools that other simulators provide i.e, for measurements, sensors, map designing or other vehicles provided with them. That would be a great idea to see if the current detection method find parking vacancy in other environment and also if this implementation of parking maneuver, our codes and calculations are not limited to Carla simulator which would ensure that this algorithm is reliable and could be prepared to use on a real street.
\section{consider traffic during parking}
In this simulation parking scenario has been implemented in a quiet street where there is no traffic , no other vehicles moving on street and no human. The only obstacles were parked vehicles. In order to make this work similar to real parking challenge on streets, it would be a good point to consider traffics of car and human during simulation.

