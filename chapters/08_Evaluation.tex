% !TeX root = ../main.tex
\chapter{Evaluation}\label{chapter:Evaluation}
This section considers functionality of the presented approach and how it works in comparison with real parking with a real driver(not automatic ones). To this end, at first the detection process (detection of vehicle and parking-bay) will be evaluated and then maneuver process and the problems during maneuvering process will be discussed.
\section{Finding parking place}
As already mentioned above, ACF method were finally selected for the detection phase as its rate on our training data was the best compared to other methods. The trained detector of this method was successful to detect our vehicles. Detection itself is done in Matlab and our main \acrshort{api} is python(PythonAPI to import Carla) so we had to make an interface vetween Python and Matlab but this connection process, made our system a little slow. using camera-sensor from Carla and waits for the Matlab detection result was the only problem we had. As it was explained in the last chapter, after each tick(or capture) of camera-sensor, vehicle should be stopped till the detection results come from Matlab file. These stops makes our vehicle a bit different from what we see in normal car parking. As we explained already, by changing the camera sensor attributes and increase the time between captures, this process can become faster but we should be careful that if this time between ticks increased a lot like more than 0.4 seconds, then some parts of the simulation that would contain a parking place would be ignored and would not be sent to the detection function. So this way would not also helpful to make a faster detection process. The other problem is that each time that camera-sensor generate a new data(JPEG file), we should connect to Matlab again which is the method used by pymatbridge. Although it is the best method for connecting that lets to access the whole .m file as once, each time we send a new data we should make a new connection to Matlab. This would be a problem for our system and also it would cause hardware problem because when there are many calls to Matlab and many connection and disconnection, it causes OS problem and system stops so for the scenario that parking vacancy is so far from the ego vehicle and many photos should be sent to Matlab before the vehicle reaches to the detected parking place, this method could not work well. So maybe it would be a good idea to change Matlab to python itself by translating detection parts from Matlab to other environment so there would not require to make these connections between Matlab and python. This idea can be considered as a future works.
\section{maneuver evaluation}
Here maneuvering phase has been evaluated and also the detailed problems and limitations to make a perfect parking motion are considered.
\subsection{Lateral Displacement of Parking-bay}
As it has been already mentioned in chapter \ref{chapter:Implementation}, the measurements to get the width of parking-bay or lateral displacement is based on the location of parked vehicles and distance between the detected vehicle and the corner side of the ego vehicle. So because of the problem at first we had with the obstacle sensor as it was just detecting Carla actors as vehicles, we could not detect the sideways or side of the parking to directly measure the distance of ego vehicle to the sideways and get lateral displacement of the parking. So what we get as a lateral distance is not really precise value which may also affect on parking motion and the calculation of formulas to get the steering and velocity magnitudes. This is assumed that by having providing more precise values, parking motion would be also improved.
\subsection{Measuring sensors' distances during maneuver}
As it has already mentioned, in order to prevent collision during movement, distances from other vehicles should be checked from obstacle sensors results. This measurement is also used to find the stop point as the sensor in the back of the vehicle compares the distance to the backward car and if it reaches the certain distance, steering angle as well as speed parameters are set to zero and the motion algorithm would be stopped. Here is a problem that the ego vehicle has not completed the parking motion yet and it is not in a parallel and perfect position compared to other adjacent cars but as the distance to the rear obstacle reached to the goal distance, vehicle should be also stopped without completing the motion so here some other forward and backward motions are required to make the perfect parking so further step of this research project would be to improve this method and make the parking algorithm as iterative \cite{parkingManeuver}. This means that in the next iterative movement direction is changed to opposite(i,e. in our case the second movement would be forward movement) and then these processes will be repeated till the car get the expected position in the parking-bay.
\subsection{Slippage and Frictions During Maneuver}
In this algorithm, frictions and slippage between vehicle's wheels and the ground are not considered and formula for steering and velocity measurement are only based on the vehicle's own parameters as steering and velocity. The problem is that in different executions even with the same scenario and no changes in the code or values, some little changes in vehicle movement can be seen like the final position in the parking place may have been a bit different from the last executions or how deep the vehicle enters to the parking in the first turn might be changed in different executions. In order to make our program more reliable in the real world, the friction values and environment situation should be also considered. This can be done by making some changes in formulas or by choosing a new maneuver algorithm.



