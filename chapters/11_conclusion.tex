% !TeX root = ../main.tex
\chapter{Conclusion}\label{chapter:Conclusion}
This study explains an implementation of parallel parking algorithm and the results of this method. As car manufactures have considered autonomous driving over the past decade, much research has been devoted to this area. Most results, however, are still hypothetical. In this thesis we specifically worked on the implementation of parallel parking and presented an implementation of parallel parking on the side of a street without line markings or defined parking spaces. Parallel parking consists of different steps: vehicle detection, parking vacancy detection, setting the vehicle starting position and finally parking maneuver to bring the vehicle into the appropriate parked position, relative to other vehicles on the street without crashes into other vehicles or surrounding obstacles. In this implementation a combination of vision- and sensor-based methods have been used. A vision based approach and machine learning algorithms were used for the detection of surrounding objects and to perform distance measurements. Although, the presented implementation was successful, it is not yet ready to apply to real vehicles on the street. Sensors should be improved to provide more precise measurement and to this goal, a study on different sensors or may be a design of a sensor which is using a variation of sensors\cite{otherThesis} should be done to get a safe and robust maneuver. In addition, parking maneuver algorithm should be improved a little as described in \ref{chapter: future_works}. 